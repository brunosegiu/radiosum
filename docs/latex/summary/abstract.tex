% !TeX spellcheck = es_ES
\begin{abstract}

Los algoritmos de iluminación global simulan el comportamiento de la luz en la naturaleza, con objetivos que involucran áreas como la síntesis de imágenes fotorealistas generadas por computadora o la evaluación del diseño lumínico para arquitectura.

En este contexto, este proyecto se enmarca en el estudio, análisis y adaptación de técnicas que proponen extensiones a el método de radiosidad. Este método se basa en el estudio de la transferencia de energía lumínica entre superficies que componen una escena. De modo de simplificar el cálculo del factor de relación geométrico se subdivide la escena en una cantidad de superficies planas discreta. Es decir, las superficies del espacio deben ser subdivididas en otras planas más pequeñas a las que llamaremos parches.

La extensión planteada propone la construcción de un algoritmo capaz de incorporar los efectos observados al considerar superficies especulares. Además de la incorporación de otras técnicas que permitan proveer nuevos acercamientos al cálculo de las relaciones que se establecen entre dos objetos en la escena que consideren las posibles optimizaciones a realizar en el hardware actual.

\end{abstract}