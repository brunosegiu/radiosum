% !TeX spellcheck = es_ES
\begin{abstract}

Los algoritmos de iluminación global simulan el comportamiento de la luz en la realidad con objetivos que involucran la síntesis de imágenes fotorealistas generadas por computadora así como la evaluación del diseño arquitectónico en lo que respecta a la iluminación.

Para ello, durante los años se han desarrollado y formalizado conceptos como la ecuación del \textit{rendering} o en lo que específicamente respecta a este proyecto la ecuación de radiosidad, pues el desarrollo de las técnicas de trazado de rayos han generado un buen nivel de resultados existen un conjunto de ventajas como la independencia de la dirección de vista que hacen de la radiosidad un método con gran potencial en los casos de uso donde las fuentes luminosas permanecen estáticas.

Esta técnica involucra la subdivisión de una escena virtual en superficies finitas conocidas como \textit{parches} a los que se les asignará un valor de radiosidad (energía lumínica) dependiente de la ubicación de las fuentes luminosas y las oclusiones entre superficies suponiendo que son reflectores lambertianos perfectos (reflectores exclusivamente difusos).

En este contexto, este proyecto se enmarca en el estudio, análisis y adaptación de técnicas propuestas a lo largo de los años que proponen extensiones a la técnica para incorporar los efectos observados al considerar superficies especulares; así como la incorporación de otras técnicas que permitan proveer nuevos acercamientos al cálculo de las relaciones que se establecen entre dos objetos en la escena.

\end{abstract}