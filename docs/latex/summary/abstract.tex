% !TeX spellcheck = es_ES
\begin{abstract}

Los algoritmos de iluminación global simulan el comportamiento de la luz en la naturaleza, con objetivos que involucran áreas como la síntesis de imágenes fotorealistas generadas por computadora o la evaluación del diseño lumínico para arquitectura.

Para ello, se han desarrollado y formalizado un conjunto de algoritmos y herramientas que resuelven el problema de forma parcial o incluso completa. Para caracterizar los problemas descriptos, se ha generado un estándar conocido como expresiones de caminos de luz \ref{LPE} (o LPEs, por sus siglas en inglés). En estas expresiones regulares se establecen distintos eventos: lumínicos o materiales. Estas expresiones son leídas de izquierda a derecha, y corresponden a un camino de la luz particular. A efectos de este proyecto, interesan los objetos \textbf{C} (cámara) y \textbf{L} (luz), así como los eventos \textbf{S} (reflexión especular),  \textbf{D} (reflexión difusa).

Los algoritmos de traza de rayos o radiosidad resuelven el problema de la iluminación global con distintos acercamentos. Orgiginalmente, el algoritmo de traza de rayos contempla los caminos de luz de la forma \textbf{L{S-D}*C}, es decir, cualquier nivel de reflexiones especulares o difusas; la solución se basa en simular haces de luz (rayos) como semi-rectas, calculando los puntos de intersección con los objetos de la escena recursivamente. Por otro lado, el algoritmo de radiosidad involucra la subdivisión de una escena virtual en superficies finitas conocidas como \textit{parches} a los que se les asignará un valor de radiosidad (energía lumínica) dependiente de la ubicación de las fuentes luminosas y las oclusiones causadas por la disposición de la geometría de la escena. Esto quiere decir, que en su concepción, el algoritmo sólo considera caminos de la forma \textbf{L{D}*C}. No obstante, es deseable que los caminos especulares sean contemplados en el cálculo de la radiosidad. 

En este contexto, este proyecto se enmarca en el estudio, análisis y adaptación de técnicas que proponen extensiones a la técnica para incorporar los efectos observados al considerar superficies especulares; así como la incorporación de otras técnicas que permitan proveer nuevos acercamientos al cálculo de las relaciones que se establecen entre dos objetos en la escena.

\end{abstract}