\chapter{Radiosidad e implementación}
\label{ch:chap02}

\section{Iluminación Global}
\label{sec:ilumglobal}

El término Iluminación Global (también conocido como Iluminación Indirecta) refiere a una modelo de
computación gráfica que simula completament interacciones de la luz con todas los objectivos virtuales tridimencionales
que se encuentran en una escena virtual. Es decir, en contraposición a la Iluminación Local, se consideran los fenómenos de
reflexión y refracción de la luz.

Por lo cual, el objetivo final de la computación obtener un valor para la radiancia de cada punto del espacio. Desde un punto de vista
matemático, todos los modelos de Iluminación Global existentes resuelven la ecuación de `rendering` de Kajiya.

Su significado se puede resumir de la siguiente manera: para calcular la radiancia de la superficie en el punto p es necesario conocer
la radiancia emitida L0 desde p al punto de vista v. Lo que es equivalente a la radiancia emitida Le adicionando la radiancia
reflejada. 

\section{Radiosidad}
\label{sec:radiosidad}

\section{Implementación de la radiosidad utilizando el método del hemicubo}
\label{sec:hemicubo}

\subsection{El método del hemicubo}

\section{Implementación de la radiosidad utilizando trazado de rayos en una hemiesfera}
\label{sec:raytracing}

\subsection{Trazando rayos en una hemiesfera}
