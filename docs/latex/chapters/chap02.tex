% !TeX spellcheck = es_ES
\chapter{Estado del arte}
\label{ch:chap02}

Este capítulo introduce un resumen de las áreas más importantes relacionadas al trabajo realizado en este proyecto, incluyendo los modelos de iluminación por computadora, el método de radiosidad y sus posibles implementaciones y extensiones.

\section{Modelos de iluminación}
\label{sec:dibujado}

El proceso de dibujado de gráficos tridimiensionales por computadora comprende la generación automática de imágenes con cierto nivel de realismo a partir de modelos que componen una \textit{escena} o \textit{mundo} tridimencional, junto a un conjunto de cualidades físicas que rigen las formas en la que la luz interactúa con los objetos.

Este problema puede ser reducido al problema de cálculo del valor de intensidad lumínica observada en un punto $x$ y proveniente directamente de un conjunto de puntos, representado por $x'$. En  \citeyear{Kajiya}, \citeauthor{Kajiya} presentó uno de los modelos más aceptado por la comunidad por su generalidad, comúnmente denominado <<ecuación del \textit{rendering}>>:

\begin{equation}
    I(x,x') = g(x,x') \bigg[\epsilon(x,x') + \int_{S} \rho(x,x',x'')I(x',x'') \delta x''\bigg] \label{eq:rendering}
\end{equation}
donde:
\begin{itemize}
    \item $I(x,x')$ describe la intensidad lumínica que llega al punto $x$ proveniente de $x'$
    \item $g(x,x')$ es un término geométrico, toma el valor de $0$ si existe oclusión entre $x'$ y $x$, y en otro caso su valor es $\dfrac{1}{r^{2}}$ donde $r$ es la distancia entre ambos puntos.
    \item $\epsilon(x,x')$ mide la energía emitida por la superficie en el punto $x'$ en dirección a $x$
    \item $\int_{S} \rho(x,x',x'')I(x',x'') \delta x''$ está compuesta por dos términos:
        \begin{itemize}
            \item $\rho(x,x',x'')$ es el término de reflectividad bi-direccional la luz que llega desde $x''$ a $x$ pasando por $x'$
            \item $I(x',x'')$ describe la intensidad lumínica observada en el punto $x'$ proveniente de $x''$
        \end{itemize}
    por lo que este término refiere a la intensidad percibida desde $x$ considerando todos las reflexiones de
    luz posibles para el dominio $S$.
\end{itemize}

Existen distintos métodos de resolución de la ecuación del rendering; la mayoría implican aproximaciones dado el gran costo de cálculo requerido para computar su valor exacto. Estos métodos balancean el costo computacional de los algoritmos utilizados y la precisión del valor obtenido. Dependiendo de las decisiones y simplificaciones consideradas existen dos clasificaciones posibles para el modelo: \textit{local} y \textit{global}, un ejemplo de ambos modelos puede ser observado en la Figura \ref{local-vs-global-img}.

\vspace{5mm}
\begin{figure}[h]
	\begin{subfigure}{0.5\textwidth}
		  	\centering
   		 	\includegraphics[width=1\linewidth]{assets/local}
   		 	\caption{Local}
   	\end{subfigure}
    \begin{subfigure}{0.5\textwidth}
    	\centering
    	\includegraphics[width=1\linewidth]{assets/global}
    	\caption{Global}
    \end{subfigure}
    \caption{Dibujado utilizando distintos modelos de iluminación}
    \label{local-vs-global-img}
\end{figure}

\subsection{Iluminación Local}
\label{sec:ilumlocal}
Los modelos de iluminación local tienen en cuenta las propiedades físicas de los materiales
y las superficies de forma individual. Es decir, al dibujar uno de los objetos no se toman en cuenta las posibles interacciones de los haces de luz con los objetos restantes en la escena. Estos métodos son frecuentemente utilizados en problemas cuya resolución debe ser realizada en tiempo real o por decisiones artísticas.

En referencia a la ecuación del rendering, el término geométrico nunca toma el valor 0, es decir, no se toma en cuenta las colisiones de la luz con otros objetos, $\epsilon(x,x')$ toma un valor constante únicamente dependiente de $x$ y $\int_{S} \rho(x,x',x'')I(x',x'') \delta x''$ toma el valor constante $0$.

\subsection{Iluminación Global}
\label{sec:ilumglobal}

El término iluminación global refiere a un modelo en donde se simulan parcial o completamente las interacciones de la luz con todos los objetos que se encuentran  en la escena. Es decir, en contraposición a la iluminación local, se consideran los fenómenos de reflexión y refracción de la luz.

Dependiendo de las característica de los modelos y algoritmos empleados, pueden obtenerse resultados más fieles a la realidad en distintos sentidos.

El algoritmo de \textit{path-tracing} emula completamente cada haz de luz desde su incepción en una fuente luminosa siguiendo el camino de interacciones del rayo con las distintas superficies de la escena. En este caso el grado de granularidad (que depende directamente de la cantidad de muestras utilizadas) impacta directamente en la precisión y calidad en la imagen final.

Por otro lado, el algoritmo de \textit{mapeado de fotones} simula los efectos producidos por las colisiones de las partículas que componen la luz (fotones) con los objetos, que dejan impresiones que afectarán el resultado final de la imágen.

Existen además distintas variaciones e híbridos de estos métodos ya que los mismos son demasiado costosos como para dibujar imágenes en tiempo real.

\section{Radiosidad}
\label{sec:radiosidad}

El método de radiosidad es una técnica de iluminación global que emula el transporte de la luz entre superficies difusas. El mismo nombre se utiliza también para describir la magnitud física definida como radiosidad, que indica el flujo de energía radiada por unidad de área ($\frac{W}{m^{2}}$).

Originalmente, este modelo de iluminación global fue propuesto por [\citeauthor{Goral}], y se basa en modelos matemáticos similares a los que resuelven el problema de la transferencia de calor en sistemas cerrados dicretos como diferencias finitas o elementos finitos.

\subsection{Radiosidad en superficies lambertianas}

La solución propuesta por \citeauthor{Goral} implica que todas las superficies son idealmente difusas, también conocidas como lambertianas. Estas superficies se comportan como reflectores difusos ideales, lo que significa que reflejan la energía incidente de forma isotrópica siguiendo la regla del coseno como se observa en la Figura \ref{img:lamber}.

\vspace{5mm}
\begin{figure}[h]
	\centering
	\includegraphics[width=1\linewidth]{assets/lambert}
	\caption{Reflector lambertiano}
	\label{img:lamber}
\end{figure}

Adicionalmente, se considerará que cada superficie irradia energía lumínica en todas direcciones en un diferencial de área $\delta_{A}$, para una dirección de vista $\omega$ puede ser definida como:

\begin{equation}
    i = \frac{\delta{P}}{\cos{\phi\delta\omega}} \label{eq:i}
\end{equation}
donde:
\begin{itemize}
    \item $i$ es la intensidad de la radiación para un punto de vista particular
    \item $\delta{P}$ es lae energía de la radiación que hemana la superficie en al dirección $\phi$ con ángulo sólido $\delta\omega$
\end{itemize}

En superficies perfectamente lambertianas, la energía reflejada puede ser expresada como: $\frac{\delta{P}}{\delta{\omega}} = k\cos{\phi}$. Donde $k$ es una constante.
Sustituyendo en \eqref{eq:i} se obtiene: $\frac{\delta{P}}{\delta{\omega}} = \frac{k\cos{\phi}}{\cos{\phi}} = k$, esto implica que la energía percibida de un punto $x$ 
es constante, independientemente del punto de vista.

Es por esto que la energía total que deja una superficie ($P$) puede ser calculada integrando la energía que deja la superficie en cada dirección posible, esto es, se integra la energía saliente en un hemi-esfera centrada en el punto estudiado:

\begin{equation}
    P = \int_{2\pi} \delta{P} = \int_{2\pi} i\cos{\phi}\delta{\omega} = i \int_{2\pi} \cos{\phi}\delta{\omega} = i\pi
    \label{eq:P}
\end{equation}

Por tanto, dada una superficie $S_{i}$, es posible calcular la energía lumínica que deja la superficie utilizando \eqref{eq:P}. Para ello, se discretizan las superficies en parches difusos, lo que transforma la Eq. \eqref{eq:P} en:

\begin{equation}
    B_{i} = E_{i} + \rho_{i} \sum_{j=1}^{N} B_{j} F_{ij} \label{eq:radiosity}
\end{equation}
donde:
\begin{itemize}
    \item $B_{i}$ es la intensidad lumínica (radiosidad) que deja la superficie $i$.
    \item $E_{i}$ es la intensidad lumínica directamente emitida por $i$.
    \item $\rho_{i}$ es la reflectividad del material para la superficie $i$.
    \item $F_{ij}$ se denomina \textit{factor de forma}, un término que representa la fracción de energía lumínica va del parche $i$ al parche $j$. 
\end{itemize}

Cabe destacar que la naturaleza recursiva de la ecuación anterior, implica que se toman en cuenta todas las reflexiones difusas que existan en la escena. Como puede observarse, resolver el sistema de $N$ ecuaciones lineales bastaría para conocer la energía emitida por cada parche. 

Los factores de emisión y reflexión, $E$ y $\mathbf{\rho}$ respectivamente, dependen de los materiales que compongan la escena, son parámetros dados. Sin embargo, resta computar la matriz de factores de forma $\mathbf{F}$ para finalmente obtener el vector de radiosidades $B$. 

Para determinar una entrada de la matriz $F_{ij}$ involucrando a las superficies $i$ y $j$ de área $A(i)$, $A(j)$, considerando los diferenciales infinitesimales de área $\delta{A_{i}}$, $\delta{A_{j}}$, representados en la Figura \ref{img:ff2}, el ángulo sólido visto por $\delta{A_{i}}$ es $\delta{\omega} = \frac{\cos{\phi_{j}\delta{A_{j}}}}{r^{2}}$. Sustituyendo en \eqref{eq:P} se obtiene:

\begin{equation}
    \delta{P}_{i}\delta{A_{i}} = i_{i} \cos{\phi_{i}}\delta{\omega}\delta{A_{i}} = \frac{P_{i}\cos{\phi_{i}}\cos{\phi_{j}}\delta{A_{i}}\delta{A_{j}}}{\pi r^{2}}
\end{equation}

\vspace{5mm}
\begin{figure}[h]
	\centering
	\includegraphics[width=0.8\linewidth]{assets/ff}
	\caption{El factor de forma entre dos superficies}
	\label{img:ff2}
\end{figure}


Considerando que ${P}_{i}{A_{i}}$ es la energía que deja $i$, y que el factor de forma $F_{ij}$ representa la fracción de dicha energía que llega a $j$ podemos observar que:

\begin{equation}
    F_{\delta{A_{i}}-\delta{A_{j}}} = \frac{\frac{P_{i}\cos{\phi_{i}}\cos{\phi_{j}}\delta{A_{i}}\delta{A_{j}}}{\pi r^{2}}}{P_{i}\delta{A_{i}}} = \frac{\cos{\phi_{i}}\cos{\phi_{j}}\delta{A_{i}}}{\pi{r^{2}}}
\end{equation}

Integrando, para obtener el factor de forma para el área total:

\begin{equation}
    F_{ij} = \frac{1}{A_{i}} \int_{A_{i}}\int_{A_{j}}\frac{\cos{\phi_{i}}\cos{\phi_{j}}\delta{A_{i}}\delta{A_{j}}}{\pi{r^{2}}} \label{eq:ff}    
\end{equation}

De \eqref{eq:ff} se obtienen las siguientes propiedades:
\begin{enumerate}
	\label{propsff}
    \item $A_{i}F_{ij} = A_{j}F{ji}$, lo que supone una relación simétrica entre los factores de forma.
    \item $\sum_{j=1}^{N} F_{ij} < 1$ Es decir, la suma de una de las filas de la matriz de factores de forma no podrá tener un valor superior a la unidad.
    \item $F_{ii} = 0$ Esto significa que el factor de forma de cada parche con respecto a sí mismo será siempre nulo.
    \item $F_{ij}$ toma el valor correspondiente a la proyección de $j$ en una hemiesfera unitaria centrada en $i$, proyectándola a su vez en un disco unitario.
\end{enumerate}


\section{Métodos de cálculo de la matriz de Factores de Forma}
\label{sec:calculoff}

El cálculo de los factores de forma a través de la Eq. \eqref{eq:ff} analíticamente es inviable en la práctica pues supone la necesidad de calcular la visibilidad entre cada par de parches que componen la escana. Por tanto, es necesario establecer otros métodos que provean aproximaciones lo suficientemente correctas.

Geométricamente, puese establecerse una analogía para la computación de factores de forma conocida como <<analogía de Nusselt>> (ver Figura \ref{img:nusselt}). Se expresará el factor de forma como la proporción de área proyectada de $S_{j}$ en una hemi-esfera ubicada en el baricentro de $S_{i}$ y luego en un disco centrado en $S_{i}$.

\begin{figure}[H]
	\centering
	\includegraphics[width=0.55\linewidth]{assets/nusselt}
	\captionof{figure}{La analogía de Nusslet}
	\label{img:nusselt}
\end{figure}

El cálculo de la matriz de factores de forma $\mathbf{F}$ supone la proyección de los parches, de aquí en más se asumirá que estos parches son polígonos no curvos y lo que permite utilizar las técnicas de dibujado de objetos tridimensionales tradicionales.

\subsection{Rasterización}
\label{sec:rasterizacion}

El <<\textit{rendering pipeline}>> es un proceso de dibujado estandarizado que consiste en un conjunto de etapas cuyo cometido es la generación de un \textit{frame buffer}. Los fabricantes de los dispositivos aceleradores gráficos y/o sistemas operativos proveen de interfaces de programación (OpenGL, Vulkan, DirectX) que se basan en este modelo para abstraer el uso del hardware.

Si bien el <<\textit{rendering pipeline}>> es modificable, cada una de sus etapas están definidas.  El programador es capaz de modificar pequeñas funciones (también llamadas \textit{kernels} o \textit{shaders}) que son ejecutadas en la GPU en las etapas correspondientes. El cometido de estas funciones es procesar los parámetros de entrada para generar parámetros que recibirá la siguiente etapa, que los recibirá y transformará como corresponda.

A continuación, se describe el proceso para OpenGL 4.5 visualizado en la Figura \ref{img:pipelinegl}, aunque muchas de estas etapas son trasladables a otras tecnologías existentes.

\begin{enumerate}
	\item Procesamiento de primitivas geométricas:
		\begin{enumerate}
			\item Especificación de vértices: Inicialmente, las aplicaciones indican un conjunto de vértices a dibujar, definiendo cierto conjunto de primitivas geométricas como triángulos, cuadriláteros, puntos, líneas u otros.
			\item \textit{Vertex shader}: Esta etapa transforma los vértices de entrada suministrados por la aplicación. Generalmente se computan las transformaciones lineales necesarias para cambiar la base de las coordenadas de los vértices de un sistema local al sistema global que defina la aplicación. Las coordenadas retornadas deberán corresponderse con coordenadas del espacio de recorte. Es decir, coordenadas correspondientes al volumen de vista.
			\item Teselado: En esta etapa se procesan los vértices a nivel de primitiva geométrica, con el objetivo de subdividirlas para mejorar la resolución obtenida.
			\item \textit{Geometry shader}: En esta etapa también se procesan los vértices a nivel de primitiva geométrica con el objetivo de mutarlas y replicarlas.
			\item Recortado: Esta etapa es \textit{fija}, es decir, no es programable. Todas las primitivas calculadas anteriormente que residan fuera del volumen de vista serán descartadas en las etapas futuras. Además, se transforma las primitivas a coordenadas de espacio de ventana.
			\item Descarte: El proceso de descarte (en inglés \textit{culling}), es también fijo. Consiste en la eliminación de primitivas que no cumplan ciertas condiciones, como por ejemplo el descarte de caras cuya normal tiene dirección opuesta a la del observador.
		\end{enumerate}
	\item Procesamiento de fragmentos (rasterización):
		\begin{enumerate}
			\item Rasterización: El proceso de rasterización discretiza las pirmitivas en espacio de pantalla en un conjunto de fragmentos.
			\item \textit{Shader de fragmentos}: El procesamiento de cada fragmento se realiza a través del \textit{shader de fragmentos} que calcula uno o más colores, un valor de profundidad, y valores de plantilla (del inglés \textit{stencil}).
			\item \textit{Scissor test}: Todos los fragmentos fuera de un área rectangular definida por la aplicación son descartados.
			\item \textit{Stencil test}: Los fragmentos que no pasan la función de planilla definida por la aplicación no son dibujados, por ejemplo, simular el \textit{scissor test} que requieran primitivas más complejas.
			\item \textit{Depth test}: En esta etapa se ejecuta el algoritmo del Z-Buffer, donde sólo se escribirá el resultado en el \textit{frame buffer} de aquellos fragmentos que tengan la menor profundidad. Es decir, los que se encuentren más cerca del observador.
		\end{enumerate}
\end{enumerate}

\clearpage

\vspace{5mm}
\begin{figure}[H]
	\centering
	\includegraphics[width=0.55\linewidth]{assets/OpenGL}
	\captionof{figure}{El \textit{rendering pipeline} de OpenGL}
	\label{img:pipelinegl}
\end{figure}

Esta técnica de dibujado es extremadamente rápida, además, la mayoría de dispositivos contienen hardware especializado capaz de acelerar estos cálculos, comúnmente conocidos como Unidades de Procesamiento Gráfico (o GPU por sus siglas en inglés). Con el objetivo de aprovechar este hardware [Cohen y Greenberg \cite{Cohen}] idearon el método del hemi-cubo para el cálculo de factores de forma.

\subsubsection{El método del hemi-cubo}

El hardware optimizado para realizar operaciones de rasterización tiene la capacidad de proyectar escenas tridimensionales en imágenes planas a gran velocidad. 

El método original de cálculo de factores de forma propone la proyección de la escena una hemiesfera centrada en $S_{i}$, sin embargo los modelos de proyección utilizados no lo permiten. Por esto es necesario proyectar la escena a un hemi-cubo centrado en $S_{i}$, esto supone el dibujado de cinco superficies planar, y por tanto puede ser realizada utilizando la rasterización.

Para utilizar el hardware eficientemente consideraremos que se calcula una fila completa de $\mathbf{F}$, esto implica que dado el parche $S_{i}$, se calcula simultáneamente los factores de forma desde $S_{i}$ al resto de las superficies restantes. 

Este método aprovecha el buffer de profundidad (Z-buffer), para la correcta determinación de visibilidad entre parches tomando en cuenta los fragmentos proyectados para los elementos que se encuentren más cercanos al parche $S_{i}$.

Este algoritmo, propuesto originalmente por [Cohen y Greenberg] en \citeyear{Cohen}, propone rasterizar la escena tridimensional en cinco texturas correspondientes a las caras de un hemi-cubo. Para cada fragmento renderizado se sumará un valor diferencial del factor de forma, que dependerá de la posición del píxel en el hemi-cubo en relación a la hemiesfera que este aproxima.  Esta suma genera una fila de la matriz $\mathbf{F}$, específicamente la fila $\mathbf{F}_{i}$, como se puede observar en la Figura \ref{img:ff}.

Por tanto, podremos definir:

\begin{equation}
	\mathbf{F}_{ij} = \sum_{q=1}^{R} \delta{F_{q}}
	\label{eq:ffgreenberg}
\end{equation}
donde:
\begin{itemize}
	\item $R$ es la cantidad de píxeles correspondientes a la superficie $S_{j}$ que cubren el hemi-cubo.
	\item $\delta{F_{q}}$ el diferencial de factor de forma asociado al píxel del hemi-cubo $q$.
\end{itemize}

\vspace{5mm}
\begin{figure}[!ht]
	\centering
	\includegraphics[width=0.8\linewidth]{assets/Hemicube}
	\captionof{figure}{Representación gráfica del método del hemicubo}
	\label{img:ff3}
\end{figure}

Los diferenciales de factores de forma deben corregir la deformación introducida con el cambio de proyección desde una hemiesfera a un hemi-cubo. Para ello, para cada píxel que compone el hemi-cubo es necesario calcular la proporición de área que este término ocupa en la hemiesfera unitaria.

Para la cara superior, los diferenciales se calculan como (ver referencias en la Figura \ref{img:deltaff}):

\begin{equation}
	\delta{F_{q}} = \frac{\cos{\phi_{i}}\cos{\phi_{j}}}{\pi{r^{2}}} \delta{A} = \frac{\delta{A}}{\pi({x^{2} + y^{2} + 1})} 
\end{equation}

Para las caras laterales, la fórmula dada es:

\begin{equation}
\delta{F_{q}} = \frac{\cos{\phi_{i}}\cos{\phi_{j}}}{\pi{r^{2}}}\delta{A} = \frac{z\delta{A}}{\pi({x^{2} + z^{2} + 1})}
\end{equation}

\begin{figure}[H]
	\centering
	\includegraphics[width=0.4\linewidth]{assets/deltaff}
	\captionof{figure}{Representación gráfica de los ejes considerados para el factor de corrección de los factores de forma}
	\label{img:deltaff}
\end{figure}


\subsection{Trazado de rayos}
\label{sec:raytracing}

Otra de las técnicas de simulación de iluminación existente es el ray tracing que consiste en el cálculo de la intersección de una semi-recta (a la que denominaremos rayo) con la geometría de la escena, (cada uno de estos rayos simulará un haz de luz.

Para cada uno de los rayos emitidos, se determinará el punto de intersección más cercano. Dada la primitiva geométrica interceptada, es posible integrar el resultado intermedio al resultado final, dependiendo del modelo de iluminación utilizado.

El trazado de rayos es una técnica efectiva [Kajiya \cite{Kajiya}] para resolver la ecuación del rendering, utilizando la técnica de \textit{trazado de camino} donde el haz de luz absorbe las propiedades de los materiales con los que interacciona. En este algoritmo, la integral se resielve con un método de Monte Carlo, donde cada rayo representa una muestra estadísticamente independiente.

\subsubsection{El método de la hemi-esfera}

El algoritmo de ray tracing puede ser utilizado par ael cálculo de factores de forma, en particular para resolver la doble integral presentada en la Eq. \eqref{eq:ff}.

Es posible re-imaginar el problema original colocando una hemi-esfera unitaria en el centro de $S_{i}$ orientada en la direccción de la normal de la superficie.

El algoritmo propuesto por \citeauthor{Malley}  consiste realizar un muestreo de la cantidad de  rayos que parten desde el centro de $S_{i}$ e intersecan $S_{j}$. Las direcciones de los rayos serán determinadas a partir de la \textit{distribución del coseno} cuya función de densidad es $f(x) = \frac{1}{2}[1 + \cos((x-1)\pi)]$.

\begin{equation}
	\mathbf{F}_{ij} = \sum_{k=1}^{nMuestras} \frac{\beta(ray(S_{i},d), S_{j})}{nMuestras} 
	\label{eq:ffhemiesfera}
\end{equation}

donde:
	
	\begin{itemize}
		\item  $d$ sigue la distribución coseno.
		\item $\beta(r, S_{x})$ toma el valor 1 si el rayo $ray(S_{i},d)$ interseca a $S_{j}$ o $0$ en otro caso.
	\end{itemize}

\vspace{5mm}
\begin{figure}[H]
	\centering
	\includegraphics[width=\linewidth]{assets/Raytracing}
	\captionof{figure}{Representación gráfica del método de método de trazado de rayos para el cálculo de factores de forma}
	\label{img:ff}
\end{figure}

Cabe destacar que, no es necesario utilizar una distribución de probabilidad con valores aleatorios o pseudo-aleatorios, sino que, si la cantidad de rayos utilizados es la suficiente y además se utiliza una función que distribuya correctamente cada rayo, los resultados obtenidos se aproximan a los reales.

Para esto, pueden utilizare otras distribuciones para la dirección de traza. Particularmente, una de ellas es la propuesta por [\citeauthor{Beckers} \cite{Beckers}], presenta un método general de teselación de discos y hemi-esferas. Es desacable el hecho de que la propuesta para hemiesferas genera un conjunto de celdas de igual área que comparten la misma relación de aspecto. Esto hace que el método presente una calidad adecuada para la elección de las direcciones en la que se trazaran los rayos.

\section{Superficies especulares}

Originalmente, el método de cálculo de la radiosidad asume que todas las superficies son reflectores lambertianos, lo que supone que solo existirán reflexiones difusas cuando la luz interactúa con ellas. Sin embargo, en la mayor parte de las escenas del mndo real es necesario simular reflexiones especulares correctamente para obtener resultados que se asemejen a la realidad.

Por ello existe la extensión del método para superficies especulares o refractantes propuesto por [\citeauthor{Sillion} \cite{Sillion}]. Los autores proponen extender el significado del término \textit{factor de forma} a más que una mera relación geométrica entre parches. El nuevo factor de forma $\mathbf{F}_{ij}$ corresponde a la proporción de energía que deja la superficie $i$ y llega la superficie $j$ luego de un número de reflexiones y refracciones especulares.

Esto modifica completamente los algoritmos de cálculo de factores de forma. Los autores proponen un algoritmo de  que cálculo consiste en el trazado de rayos desde $S_{i}$ en una dirección arbitraria $d$ bien distribuida.  Luego, una vez que se conozca el camino trazado se distribuirá el valor final del factor de forma dependiendo en la cantidad de superficies con las que interaccione el rayo y sus coeficientes especulares como se observa en la Figura \ref{img:caminoespecular}.

\vspace{5mm}
\begin{figure}[H]
	\includegraphics[width=1\linewidth]{assets/extended}
	\captionof{figure}{Representación gráfica del cálculo del factor de forma extendido donde $k = \frac{1}{N}$, con $N$ muestras tomadas.}
	\label{img:caminoespecular}
\end{figure}

\section{Cálculo del vector de radiosidades}
\label{sec:vrad}

Luego de computar la matriz $\mathbf{F}$ y dado los vectores de emisiones $E$ y reflexiones $\rho$, resta computar el vector de radiosidades correspondiente para cada parche, denominado $B$.

Recordando la Eq. \eqref{eq:radiosity}, es posible deducir el problema al sistema de ecuaciones dado por:

\begin{equation}
	E = (\mathbf{I} - \mathbf{RF})B
\end{equation}

Los estudios de álgebra lineal modernos permiten la resolución de sistemas de ecuaciones de forma optimizada, dependiendo de las propiedades observadas.

Recordando las propiedades en la Sección \ref{propsff}, podemos observar que:

\begin{itemize}
	\item $\sum_{j=1}^{N} \mathbf{F}_{ij} \leq 1 \forall{i \in [1,N]}$
	\item $\rho_{i} \leq 1 \rightarrow \sum_{j=1}^{N} \mathbf{R}_{ij} \leq 1 \forall{i \in [1,N]}$
\end{itemize}

Esto implica que las entradas de $\mathbf{RF}$ s\texttt{}on siempre menores a $1$, por tanto la matriz $(\mathbf{I} - \mathbf{RF}) = M$ es diagonal dominante ya que $\sum_{j=1}^{N}|R_{ij}F_{ij}| \le 1 \forall i \in [1, N]$ y $R_{ii}F_{ii} = 0  \forall  i \in [1,N]$. Esto garantiza la convergencia del uso de métodos de resolución iterativos o de factorización, como el algoritmo de Gauss-Seidel o la factorización LU.

Aunque los algoritmos clásicos de resolución de sistema de ecuaciones aplican a este problema, existen optimizaciones que hacen que su resolución se pueda aproximar de manera razonable con un costo computacional muy menor. Para ello, considerando que la matriz $\mathbf{F}$ es diagonal dominante, podemos utilizar la equación \ref{eq:iterativo} pues el \textit{residuo} (el término agregado en cada iteración) se reduce de la siguiente forma: $\left\|\mathbf{RF}B^{(i+1)}\right\| < \left\|\mathbf{RF}B^{(i)}\right\|$. El método planteado en la Eq. \eqref{eq:iterativo} es el método de Jacobi.

\begin{equation}
	B^{(i+1)}  = \mathbf{RF}B^{(i)}  + E \text{ con }  B^{(0)} = E
	\label{eq:iterativo}
\end{equation}

Cabe aclarar, que el método planteado hasta el momento resuelve la radiosidad en un único canal. Es decir, no se toma en cuenta todo el espectro electromagnético de la luz, es por ello que puede establecerse una extensión del método. Esta extensión implica la existencia de tres vectores de reflexión, uno para cada canal \textit{RGB} (del inglés \textit{Red - Green - Blue}). Por tanto es necesario que se resuelvan tres y no un único sistema de ecuaciones, aunque es posible destacar que la matriz $\mathbf{F}$ permanece constante pues depende de la geometría de la escena. El único cambio en el sistema surge en la matriz $\mathbf{R}$ que pasará a depender del canal seleccionado: $\mathbf{R}_{c}$.

\section{OpenGL}

Con el objetivo de proveer interfaces estandarizadas para el uso de tarjetas gráficas y los distintos algoritmos relacioandos a la rasterización existen un conjunto de interfaces que abstraen los recursos necesarios (hardware, sistema operativo). En el contexto de este proyecto, se estudia el uso de Open Graphics Library (OpenGL).

OpenGL es una especificación de una Interfaz de Programación de Aplicación (API por sus siglas en inglés) diseñada por la organizción Khronos Group. Su cometido es el dibujado de gráficos bidimensionales o tridimensionales utilizando el método de rasterización (ver Sección \ref{sec:rasterizacion}). Los distintos fabricantes de Sistemas Operativos y tarjetas gráficas proporcionan implementaciones que se ajusten al hardware específico. Esta abstracción facilita la compatibilidad de las aplicaciones independientemente del hardware donde sean ejecutadas.

\subsection{Arquitectura}
La arquitectura base de la bibliotecta es de cliente/servidor (ver Figura \ref{img:gpucpugl}). El cliente es la aplicación que invoca funciones para el dibujado de gráficos y es ejecutado en la CPU. El servidor, que es ejecutado en la GPU, almacena los distintos buffers y ejecuta las funciones necesarias.

El cliente modifica los atributos a través de invocaciones a las funciones de prefijo \verb|gl|, identificando el recurso afectado con valores enumerados (por ejemplo, \verb|GL_TEXTURE_2D| representa un conjunto de imágenes bidimencionales). Dado que la biblioteca es implementada como una máquina de estado, los atributos son recordados hasta que sean modificados nuevamente.

Estas invocaciones no son ejecutadas inmediatamente, sino que de forma similar a un buffer de entrada/salida son almacenados para ser ejecutados cuando sea necesario, es decir, cuando se requiera el dibujo de una nueva imagen. Esto hace que la ejecución de comandos sea asíncrona, y por tanto mejora el rendimiento previniendo la sincronización entre la CPU y GPU.

\vspace{5mm}
\begin{figure}[h]
	\centering
	\includegraphics[width=.4\linewidth]{assets/cpu_gpu}
	\caption{Vista general de la arquitectura de OpenGL}
	\label{img:gpucpugl}
 \end{figure}

\subsection{Extensiones}
La inicialización de la máquina de estados depende directamente de la creación de un contexto que será utilizado para almacenar los datos. Este proceso depende fuertemente de la plataforma donde se ejecute la aplicación, que depende entre otros del sistema operativo y/o el hardware utilizado. Por este motivo, existen bibliotecas que manejan la creación del contexto en diversas plataformas como SDL y GLFW.
	
\section{Embree}

Los distintos algoritmos para evaluar la intersección entre superficies y rayos han evolucionado a gran velocidad, introduciéndose los conceptos de volumen envolvente y jerarquías de escena. Esto resulta en un gran re-trabajo al momento de implementar algoritmos que se basan en el trazado de rayos de forma eficiente. Es por ello, que de manera similar a las APIs de dibujado de gráficos acelerados por hardware existen interfaces que facilitan la aceleración del trazado de rayos. En particular, Embree es una biblioteca creada por Intel con este propósito.

La biblioteca expone un conjunto de funciones para realizar el trazado de rayos acelerado a través de componentes de hardware y software mediante la utilización del conjunto de instrucciones del paradigma SIMD (del inglés Single Instruction - Multiple Data), donde una única instrucción es ejecutada sobre un gran conjunto de datos (por ejemplo, la ejecución concurrente de un conjunto de multiplicaciones en punto flotante a nivel de CPU) y la generación de estructuras de aceleración, como las BVH (del inglés \textit{Bounding Volume Hierarchies}). La arquitectura de la aplicación, diagramada en la Figura \ref{img:embree}, demuestra los distintos algoritmos propuestos para la generación de estructuras de aceleración y algoritmos de intersección eficientes.

La biblioteca resuelve un conjunto de dificultades normalmente encontradas en todas las aplicaciones de algoritmos que involucren el trazado de rayos, entre ellas:

\begin{itemize}
	\item \textbf{Multi-hilo:} Con el objetivo de ejecutar distintos kernels de traza de rayos de forma concurrente, la biblioteca provee de funciones \textit{thread-safe} para el dibujado y la generación de estructuras de aceleración.
	\item \textbf{Vectorización:} Con el objetivo de optimizar el uso de la CPU, la biblioteca vectoriza los cálculos necesarios para aprovechar las instrucciones SIMD.
	\item \textbf{Soporte para múltiples CPUs:} La biblioteca provee de una capa de abstracción independiente del hardware donde se utilice.
	\item \textbf{Conocimiento del dominio extenso:} Dado que la biblioteca implementa las estructuras de aceleración y los algoritmos de intersección no es necesario tener un conocimiento completo del dominio para construir aplicaciones utilizando trazado de rayos.
	\item \textbf{Manejo eficiente de la memoria:} Para la visualización de escenas con gran cantidad de primitivas.
\end{itemize}

\vspace{5mm}
\begin{figure}[H]
	\centering
	\includegraphics[width=.8\linewidth]{assets/embree}
	\caption{Vista general de la arquitectura de Embree}
	\label{img:embree}
\end{figure}

\section{Trabajos relacionados}

En esta sección se discuten alternativas propuestas para resolver el cálculo de la iluminación global en escenas con materiales difusos y especulares.

\subsection{Un método de trazado de rayos para el cálculo de iluminación es escenas difusas-especulares}

El algoritmo propuesto por \citeauthor{Shirley} en \citeyear{Shirley} calcula la iluminación difusa utilizando dos pasadas. Este algoritmo difiere del propuesto por \citeauthor{Sillion} en el sentido que se consideran distintos modelos de fuentes luminosas con propiedades particulares (luces puntuales, direccionales, de área).

En la primer pasada, el algoritmo calcula la componente difusa de todos los rayos que rebotan en al menos una superficie especular utilizando el método de Arvo. Este método considera calcula los caminos que seguirán los rayos de luz provenientes de fuentes luminosas, es decir, se discretiza la cantidad de rayos emitidos por una fuente luminosa, cada rayo representa una fracción de la energía emitida.

Cuando existe una intersección, se divide la energía entre los cuatro nodos más cercanos (estos nodos almacenan la radiosidad) a través de una estimación para calcular qué área ocupa cada uno de ellos, de esta manera es posible generar un mapa de radiosidad para la superficie. Dado que la iluminación directa (es decir, aquellos rayos que no se intersecan con superficies especulares) es calculada en la etapa de vista, solo es necesario computar los rebotes especulares, para ello se traza un número bajo de rayos distribuidos de forma uniforme para encontrar las zonas donde existan superficies especulares, luego se trazan rayos en esa dirección de forma "densa", que implica trazar una cantidad de rayos considerable en una dirección que no varía demasiado.

El segundo paso utiliza el método de radiosidad para calcular la iluminación difusa que involucra al menos dos superficies, nuevamente se omite la iluminación directa pues se calculará en la etapa de vista. Para ello, se emiten rayos desde cada supeficie utilizando la distribución del coseno de manera equivalente a la propuesta por \citeauthor{Malley}. 

Finalmente, cuando se dibuja la imagen final también se calcula la iluminación directa de forma estándar (ver \cite{Whitted}) sustituyendo el término de ambiente por el calculado en las pasadas anteriores.

\subsection{Iluminación especular rápida incluyendo efectos especulares}

Otro acercamiento a la emulación de efectos de reflexiones especulares implica el uso de sistemas de partículas o fotones para seguir el camino de la luz cuando interacciona con este tipo de superficies. El algoritmo propuesto por \citeauthor{Granier} implica la construcción de un grafo jerárquico representando la geometría de la escena, cada nodo del árbol representa un subespacio donde se calculará la radiosidad, el refinamiento (cantidad de niveles) incide directamente en el error observado.

Al observar al transferencia de luz entre los nodos que componen la escena, las transferencias difusas-especulares son computadas a través de trazado de fotones entre el subconjunto de objetos seleccionado. Esto significa que las partículas dejarán trazas de energía dependiendo del tipo de superficie donde incidan, además rebotarán en función de distribución asignada a la superficie.

\subsection{Un método de dos pasadas para el cálculo de radiosidad en parches de Bezier}

El método propuesto por \citeauthor{Kok} es una extensión para parches que están formados por superficies de Bézier, estas son superficies delimitadas por curvas de nombre homónimo que para una superficie definida con $m$ puntos siguen la ecuación $c(u,v) = \sum_{i=0}^{m} c_{i}(v)B_{i}^{m}(u)$ donde $c$ es el vector de desplazamientos, y $B$ una función que genera la curva.

Los autores proponen la discretización de las superficies en puntos de muestreo dependiendo del área, luego simplemente se calcula el factor de forma de la superficie utilizando el método de la hemi-esfera, agregando los resultados para cada punto. En caso de que un rayo interseque una superficie especular, los autores proponen un método similar al de \citeauthor{Sillion} donde se seguirá el camino del rayo mientras rebote en superficies especulares y arribe en una difusa, distribuyendo el factor de forma entre las superficies involucradas dependiendo del coeficiente de reflexión especular.